\section{Auswertung}
\label{sec:Auswertung}
\subsection{Bestimmung der Schallgeschwindigkeit}

Zur Berechnung der Schallgeschwindigkeit $c_\text{Schall}$ mittels der Impuls-Echo-Methode werden die Daten in \autoref{tab:impuls-echo} geplottet.
Über diese Werte wird ein linearer Fit der Form

\begin{center}
    $d = c_\text{Schall,1} \cdot t + x_{0,1}$
\end{center}

gelegt. Dabei ist zu beachten, dass $d$ den doppelten Betrag der Länge der Zylinder hat, da der Schall erst zu der Rückwand dieser und dann zurück zum Detektor laufen muss.
Der entstehende Graph ist in \autoref{fig:plot_schall_impuls} zu sehen.

\begin{table}[!htp]
\centering
\caption{Daten der Durchschallung von verschiedenen Acrylzylindern mit der Impuls-Echo-Methode.}
\label{tab:impuls-echo}
\begin{tabular}{S[table-format=2.2] S[table-format=1.2] S[table-format=1.1] S[table-format=1.2] S[table-format=2.2] S[table-format=2.2]}
\toprule
{$d$ / cm} & {$U_1$ / V} & {$t_1$ / µs} & {$U_2$ / V} & {$t_2$ / µs} & {$\Delta t$ / µs} \\
\midrule
 3.11 & 1.37 & 0.7 & 1.37 & 23.4 & 22.7 \\
 4.05 & 1.36 & 0.7 & 1.36 & 30.2 & 29.5 \\
 8.05 & 1.36 & 0.7 & 1.32 & 59.6 & 58.9 \\
10.21 & 1.37 & 0.7 & 0.46 & 76.4 & 75.7 \\
12.05 & 1.36 & 0.7 & 0.90 & 88.8 & 88.1 \\
16.10 & 1.36 & 0.7 & 1.34 & 30.2 & 29.5 \\
18.26 & 1.35 & 0.8 & 0.55 & 76.4 & 75.6 \\
\bottomrule
\end{tabular}
\end{table}

\begin{figure}
    \centering
    \includegraphics[width=0.9\textwidth]{build/plot_schall_impuls.pdf}
    \caption{Plot und Fit der Messwerte der Impuls-Echo-Methode zur Bestimmung der Schallgeschwindigkeit, wobei die zwei letzten Messwerte nicht betrachtet werden.}
    \label{fig:plot_schall_impuls}
\end{figure}

Zur Berechnung der Koeffizienten werden die letzten beiden Werte nicht betrachtet, da diese eine ungewöhnliche und sehr große Abweichung von der Geraden aufweisen.
Somit werden jene mittels Python 3.7.0 als

\begin{center}
    $c_\text{Schall,1} = (0,271 \pm 0,002) \frac{\symup{cm}}{\symup{µs}} = (2,71 \pm 0,02) \cdot 10^{3} \frac{\symup{m}}{\symup{s}}$
    
    $x_{0,1} = (0,1 \pm 0,1)$ cm
\end{center}

ermittelt.

%%%%%%%%%%%%%%%%%%%%%

\begin{table}[!htp]
\centering
\caption{Daten der Durchschallung von verschiedenen Acrylzylindern mit der Durchschallmethode.}
\label{tab:durchschall}
\begin{tabular}{S[table-format=2.2] S[table-format=2.1] S[table-format=1.1]}
\toprule
{$d$ / cm} & {$t$ / µs} & {$U$ / V} \\
\midrule
 3.11 & 12.6 & 0.3 \\
 4.05 & 15.5 & 0.3 \\
 8.05 & 30.3 & 0.2 \\
10.21 & 38.6 & 0.2 \\
12.05 & 45.4 & 0.2 \\
\bottomrule
\end{tabular}
\end{table}

Zur Bestimmung der Schallgeschwindigkeit mit der Durchschall-Methode wird analog zu der Impuls-Echo-Methode vorgegangen:
Es werden die Daten aus \autoref{tab:durchschall} geplottet und über diese wird erneut ein linearer Fit der Form

\begin{center}
    $d = c_\text{Schall,2} \cdot t + x_{0,2}$
\end{center}

Hier entspricht $d$ genau der Länge der Zylinder. Der entsprechende Plot ist in \autoref{fig:plot_schall_durch} zu finden.

\begin{figure}
    \centering
    \includegraphics[width=0.9\textwidth]{build/plot_schall_durch.pdf}
    \caption{Plot und Fit der Messwerte der Durchschall-Methode.}
    \label{fig:plot_schall_durch}
\end{figure}

Ebenfalls mittels Python 3.7.0 werden die Koeffizienten ermittelt:

\begin{center}
    $c_\text{Schall,2} = (0,271 \pm 0,003) \frac{\symup{cm}}{\symup{µs}} = (2,71 \pm 0,03) \cdot 10^{3} \frac{\symup{m}}{\symup{s}}$
    
    $x_{0,2} = (-0,21 \pm 0,08)$ cm.
\end{center}


\subsection{Bestimmung des Absorptionskoeffizienten}

Zunächst wird aus GLEICHUNG die akustische Impedanz als

\begin{center}
    $Z \approx 3,28 \cdot 10^{6}$ $\frac{\symup{kg}}{\symup{s}\symup{m}^2}$
\end{center}

ermittelt. Dazu wird $\rho = 1190$ $\frac{\symup{kg}}{\symup{m}^3}$ \cite{pmma} genutzt. 
Mit dem Wert für die akustische Impedanz von Luft $Z_\text{Luft} \approx 413$ $\frac{\symup{kg}}{\symup{m}^3}$\cite{akImLuft} lässt sich anhand von GLEICHUNG der Reflexionskoeffizient $R \approx 1$ bestimmen.
Daher wird im Folgenden davon ausgegangen, dass der gesamte Schall reflektiert wird.

Zur Bestimmung des Absorptionskoeffizienten $\alpha$ werden die Werte aus \autoref{tab:impuls-echo} geplottet.
Dabei wird die $U(x)$-Achse logarithmisiert, um so eine Gerade zu erhalten.
Über diese wird ein Fit der Form

\begin{center}
    $\ln \bigg( \frac{U}{1 \symup{V}} \bigg) = \ln \bigg( \frac{U_0}{1 \symup{V}} \bigg) - \alpha x$
\end{center}

gelegt. Erneut ist darauf zu achten, dass $d$ den doppelten Betrag der Länge der Zylinder hat.
Der so entstehende Plot ist in \autoref{fig:plot_daempfung} zu sehen.

\begin{figure}
    \centering
    \includegraphics[width=0.9\textwidth]{build/plot_daempfung.pdf}
    \caption{Plot und Fit der Messwerte der Impuls-Echo-Methode zur Bestimmung des Absorptionskoeffizienten, wobei zwei Messwerte nicht betrachtet werden.}
    \label{fig:plot_daempfung}
\end{figure}

Aufgrund starker Abweichungen von zwei Werten, werden diese in der Ausgleichsrechnung nicht näher betrachtet.
Somit ermitteln sich die Werte mittels Python 3.7.0 als

\begin{center}
    $\ln \bigg( \frac{U_0}{1\symup{V}} \bigg) = (0,6 \pm 0,1)$

    $\alpha = (0,031 \pm 0,005) \frac{1}{\symup{cm}} = (3,1 \pm 0,5) \frac{1}{\symup{m}}$
\end{center}

bestimmt.

\subsection{Biometrische Abmessung eines Augenmodells}

Bei der Vermessung des Augenmodells werden mehrere ausgeprägte Peaks gemessen. Die Daten sind in \autoref{tab:auge} zu finden.
Der nullte bei $t \approx 0$ s entseht an der Grenzfläche Sonde-Hornhaut. Die weiteren relevanten Peaks sind Peak Nummer 1 (Augenkammer-Linse), 2 (Linse-Glaskörper) und 5 (Retina).
Peak Nummer 3 und 4 entstehen durch Reflexionen im Auge selbst, beispielsweise durch zweimalige Reflexionen in der Linse selbst.

\begin{table}[!htp]
\centering
\caption{Zeitabstände der detektierten Pulse bei der Durchschallung eines Augenmodells mit der Impuls-Echo-Methode.}
\label{tab:auge}
\begin{tabular}{S[table-format=1.0] S[table-format=2.1]}
\toprule
{Peak Nr} & {$t$ / µs} \\
\midrule
1 & 15.8 \\
2 & 23.5 \\
3 & 32.3 \\
4 & 38.6 \\
5 & 72.9 \\
\bottomrule
\end{tabular}
\end{table}

Die Schallgeschwindigkeiten innerhalb der verschiedenen Materialien variieren. Diese werden aus der Literatur entnommen:

\begin{center}
    $c_\text{Augenkammer} = 1532$ $\frac{\symup{m}}{\symup{s}}$\cite[63]{augeZahlen}

    $c_\text{Linse} = 2500$ $\frac{\symup{m}}{\symup{s}}$ \cite{US1}

    $c_\text{Glaskörper} = 1410$ $\frac{\symup{m}}{\symup{s}}$ \cite{US1}.
\end{center}

Damit lassen sich die Distanzen berechnen wie folgt bestimmen:

\begin{center}
    Hornhaut - Anfang Linse: $s_1 = \frac{1}{2} c_\text{Augenkammer} t_1 = 1,21$ cm,

    Hornhaut - Ende Linse: $s_2 = \frac{1}{2} c_\text{Linse} + s_1 = 4,14$ cm

    und

    Hornhaut - Retina: $s_3 = \frac{1}{2} c_\text{Glaskörper} + s_2 = 9,29$ cm.
\end{center}