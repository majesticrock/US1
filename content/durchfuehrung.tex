\section{Durchführung}
\label{sec:Durchführung}
Im ersten Versuchsteil werden fünf verschieden große 
Zylinder mittels Impuls-Echo-Verfahren vermessen.
Dabei soll die Schallgeschwindigkeit von Acryl 
bestimmt werden.
Hierzu wird zunächst eine Ultraschallsonde mit bidestilliertem 
Wasser an die Acrylzylinder gekoppelt. Ein Kontaktmittel ist
notwendig, da Ultraschall sehr stark von Luft absorbiert wird.
Es werden die Amplituden, sowie die Laufzeiten der ersten
beiden reflektierten Pulse bestimmt und aufgenommen.
Zusätzlich wird die Länge der Acrylzylinder mit einer Schieblehre 
bestimmt. 
Des Weiteren lässt sich aus den Daten des zweiten gemessenen
Impulses die Dämpfung bestimmen.
Zur Weiteren Bestimmung der Schallgeschwindigkeit in Acryl
wird die vorrangegangene Messung für zwei weitere Zylinderlängen
durchgeführt, wobei die zu messenden Zylinder nun aus 
zwei Zylindern zusammengesetzt werden. 


Anschließend wird die Messung der ersten fünf Zylinder 
ein weiteres Mal durchgeführt, wobei dieses Mal das 
Durchschallungsverfahren angewandt wird. Dazu werden die 
Zylinder und die Ultraschallsonden jeweils in eine Halterung
gebracht und mit Koppelgel angekoppelt. Neben den mit einer
Schieblehre zu bestimmenden Längen der Zylinder sind die Laufzeiten 
der Ultraschallimpulse aufzunehmen.


Im letzten Versuchsteil wird ein Augenmodell mithilfe der Ultraschallsonde
durch das Impuls-Echo-Verfahren vermessen. Dazu wird 
die Ultraschallsonde mit Koppelgel auf die Mitte der Hornhaut
gesetzt. Der Einschallwinkel wird jetzt so gewählt, dass 
das Echo der Retina zu sehen ist. Die Laufzeiten der Echso
an Iris und Retina werden aufgenommen, woraus sich die Abmessungen 
des Auges bestimmen lassen.
