\section{Diskussion}
\label{sec:Diskussion}

%Der Versuch ist ein Fehler!

Die errechneten Werte für die Schallgeschwindigkeit von $c_\text{Schall,1} = (2,71 \ pm 0,02) \cdot 10^{3}$ $\frac{\symup{m}}{\symup{s}}$ und $c_\text{Schall,2} = (2,71 \ pm 0,03) \cdot 10^{3}$ $\frac{\symup{m}}{\symup{s}}$
liegen beide im Rahmen der Literaturwerte von $c_\text{Acryl} = 2670$ bis $2760$ $\frac{\symup{m}}{\symup{s}}$\cite{acryl}.
Da beide Werte auch eine ähnliche Standardabweichung aufweisen, scheinen beide Methode gute Werte zu liefern. Dies ist auch gut an den Plots zu sehen:
Die Messwerte liegen ziemlich genau auf dem jeweiligen Fit.
Es ist jedoch zu beachten, dass die gestapelten Zylinder bei der Impuls-Echo-Methode offensichtlich fehlerhafte Werte geliefert haben, welche in der Rechnung nicht beachtet werden dürfen.

Zur Bestimmung des Abschwächungskoeffizieten $\alpha = (3,1 \pm 0,5) \frac{1}{\symup{m}}$ ist zu bemerken, dass einige Werte ebenfalls weit von der Erwartung abweichen. Diese wurden daher bei der Berechnung nicht beachtet.
Jedoch auch die beachteten Werte weichen verglichen mit der Schallgeschwindigkeitsmessung stark von der Ausgleichsgeraden ab.
Daraus entsteht die die vergleichsweise hohe Standardabweichung von circa $16,1 \%$.

Die Messung selbst gestaltet sich als einfach, da ein statisches System betrachtet wird.

Bei der biometrischen Ausmessung des Augenmodells fällt jedoch auf, dass selbst kleine Veränderungen im Einfallswinkel des Schalls die Form des Graphen stark verändern können.
Ebenfalls wirken sich Reflexionen im Inneren des Modells auf die Form dessen aus. Das Gesamtbild ergibt sich dennoch realitisch mit den äußeren Abmessungen des Modells.