\section{Zielsetzung}
    Ziel dieses Versuches ist es die Schallgeschwindigkeit in Acryl mittels einer Ultraschallmessung zu bestimmen. Außerdem wird ein Augenmodell
    mittels Ultraschall vermessen.
\section{Theorie}
\label{sec:Theorie}
    Schallwellen sind longitudinale Wellen, welche sich durch Druckschwankungen in Medien fortbewegen. Diese Schwankung kann durch
    \begin{equation}
    \label{eqn:welle}
        p(x,t) = p_{\symup{0}} + v_{\symup{0}} Z \cos{\omega t - k x}
    \end{equation}
    beschrieben werden, wobei 
    \begin{equation}
    \label{eqn:impedanz}
        Z = \rho \cdot c
    \end{equation}
    die akustische Impedanz ist. Dabei ist $\rho$ die 
    Dichte des Mediums durch das sich der Schall bewegt
    und $c$ die zugehörige Schallgeschwindigkeit des
    Materials. Hieran wird deutlich, dass Schallwellen im 
    Vergleich zu elektromagnetischen Wellen materialabhängig ist.
    Andere Eigenschaften elektromagnetischer Wellen, wie zum Beispiel
    Reflexion und Beugung werden aber auch von Schallwellen 
    erfüllt. Die Schallgeschwindigkeit in Festkörpern
    ist über das Elastizitätsmodul $E$ und die Dichte 
    $\rho$ zu 
    \begin{equation}
    \label{eqn:schallgeschw_festkoerper}
        c_{\symup{Fe}} = \sqrt{\frac{E}{\rho}}
    \end{equation}
    definiert. Zu Beachten ist hier aber, dass sich der 
    Schall in Festkörpern nicht nur longitudinal, sondern 
    als Folge von Schubspannungen auch transversal ausbreitet.
    Des Weiteren nimmt die Intensität $I_{\symup{0}}$ des Schalls bei Ausbreitung
    exponentiell in der Strecke $x$ ab:  
    \begin{equation}
    \label{eqn:abnahme}     
        I(x) = I_{\symup{0}} \cdot e^{-\alpha x}.
    \end{equation}
    Dabei ist $\alpha$ der Absorptionskoeffizient
    der Schallamplitude. 
    Beim Auftreffen des Schalls an einer Grenzfläche wird 
    ein Teil des Schalls reflektiert. Es wird daher ein 
    Reflexionskoeffizient $R$ als das Verhältnis von 
    reflektierter zu eintreffender Schallintensität definiert.
    Der Reflexionskoeffizient
    \begin{equation}
    \label{eqn:reflexionskoeffizient}
    R = \Bigr ( \frac{Z_{\symup{1}} - Z_{\symup{2}}} {Z_{\symup{1}} + Z_{\symup{2}}}   \Bigl)^2
    \end{equation}
    ist dabei Abhängig von den akustischen Impedanzen $Z_{\symup{i}}$
    der beiden angrenzenden Materialien.
    Des Weiteren ist der transmittierte Anteil $T$ durch
    \begin{equation}
    \label{eqn:transmission}
        T = 1 - R 
    \end{equation}
    bestimmt.

    Zur Erzeugung von Ultraschall kann der sogenannte piezo-elektrische Effekt 
    genutzt werden. Dazu wird ein piezoelektrischer Kristall in ein elektrisches 
    Wechselfeld gebracht und wenn eine polare Achse des Kristalls in Richtung 
    des elektrischen Feldes zeigt, wird dieser dadurch zu Schwingungen angeregt,
    wobei Ultraschall abgestrahlt wird. Falls die Ausgangsfrequenz mit der 
    Eigenfrequenz des Kristalls übereinstimmt kommt es zur Resonanz und es können
    hohe Energiedichten erzeugt werden. Umgekehrt kann der Kristall auch als 
    Empfänger für Ultraschall genutzt werden. Dabei treffen Schallwellen auf den 
    Kristall und regen diesen zu Schwingungen an, die dann aufgenommen werden 
    können. 
    Allgemein wird hier zwischen zwei Arten der Ultraschallmessung unterschieden.
    Es gibt einmal das Durchschallungsverfahren, bei dem mit einem 
    Ultraschallsender ein Schallimpuls ausgesendet wird und am anderen 
    Ende der Probe durch einen Empfänger gemessenw wird. Falls eine Fehlstelle
    in der Probe existiert so wird das durch eine geringere Intensität deutlich.
    Mit dieser Methode ist es allerdings nicht möglich Aussagen über 
    den Ort der Fehlstelle zu treffen.
    Die zweite Methode ist das Impuls-Echo-Verfahren, bei dem 
    der Ultraschallsender auch als Empfänger genutzt wird. 
    Der ausgesendete Ultraschallimpuls wird an Grenzflächen reflektiert
    und vom Empfänger aufgenommen. Diese Methode erlaubt anderes als die vorherige 
    Aussagen über die Position von Fehlstellen zu treffen.
    Wenn die Schallgeschwindigkeit im Material bekannt ist kann die Position der 
    Fehlstelle über 
    \begin{equation}
    \label{eqn:fehlstelle}
        s=\frac{1}{2} c t
    \end{equation}
    bestimmt werden.

